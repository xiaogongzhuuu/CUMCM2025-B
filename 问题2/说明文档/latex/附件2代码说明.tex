\section{SiC 15° 入射角外延层厚度计算}

\subsection{研究目的}
在碳化硅 (SiC) 外延层厚度测量中,红外干涉法是一种常用的非破坏性手段。由于外延层与衬底之间存在折射率差异,在一定入射角下会产生干涉条纹。通过提取条纹间隔并结合色散模型,可以反推出外延层厚度。本文以 \SI{15}{\degree} 入射角实验数据为例,基于 \textbf{Cauchy+Drude 模型} 拟合折射率并进行厚度计算。

\subsection{数学模型与物理基础}

\subsubsection{菲涅尔反射率}
在入射角 $\theta_i$ 下,反射率由菲涅尔公式计算:
\begin{equation}
R = \frac{1}{2}\left( 
\left|\frac{\cos\theta_i - n \cos\theta_t}{\cos\theta_i + n \cos\theta_t}\right|^2 
+ 
\left|\frac{n\cos\theta_i - \cos\theta_t}{n\cos\theta_i + \cos\theta_t}\right|^2 
\right),
\end{equation}
其中,折射角 $\theta_t$ 由 Snell 定律:
\begin{equation}
\sin\theta_t = \frac{\sin\theta_i}{n}.
\end{equation}

\subsubsection{折射率模型:Cauchy+Drude}
外延层折射率 $n(\lambda)$ 的平方形式为:
\begin{equation}
n^2(\lambda) = \left(A + \frac{B}{\lambda^2} + \frac{C}{\lambda^4}\right)^2 - \frac{Ne^2}{\varepsilon_0 m^\ast \omega^2},
\end{equation}
其中:
\begin{itemize}
    \item $(A, B, C)$ 为 Cauchy 色散系数;
    \item $N$ 为自由载流子浓度;
    \item Drude 模型部分描述了自由电子对红外折射率的修正。
\end{itemize}

\subsubsection{厚度公式}
由干涉条件可得外延层厚度:
\begin{equation}
d = \frac{\lambda \, (\lambda + \Delta \lambda)}{2 \, n(\lambda) \cos\theta_t \, \Delta \lambda},
\end{equation}
其中 $\lambda$ 为中心条纹波长,$\Delta \lambda$ 为平均条纹间隔。

\subsection{程序实现步骤}

\begin{enumerate}
    \item \textbf{数据读取与预处理}  
    从 \texttt{附件2.xlsx} 读取实验反射率数据;转换波数为波长(\si{\micro\meter});使用 Savitzky–Golay 滤波器平滑曲线,抑制高频噪声。

    \item \textbf{折射率反演与拟合}  
    根据菲涅尔公式,从实验反射率反演折射率;对有效数据区间,拟合 Cauchy 模型参数 $(A, B, C)$;在此基础上,通过最小化实验反射率与模型反射率之间的误差,反演最佳载流子浓度 $N$,得到完整的 Cauchy+Drude 模型。

    \item \textbf{条纹提取与厚度计算}  
    使用 \texttt{find\_peaks} 提取干涉峰位置;计算相邻条纹间隔 $\Delta \lambda$;选择中间条纹波长作为代表值,计算折射率 $n(\lambda)$ 与折射角 $\theta_t$;利用严格公式计算厚度 $d$。

    \item \textbf{结果输出}  
    程序自动生成:
    \begin{itemize}
        \item \texttt{波长\_折射率\_15度.xlsx}:拟合折射率数据;
        \item \texttt{条纹间隔\_15度.xlsx}:干涉条纹波长与间隔;
        \item \texttt{厚度计算结果\_15度.txt}:外延层厚度及条纹参数;
        \item \texttt{SiC\_15度拟合.png}:拟合结果可视化。
    \end{itemize}
\end{enumerate}

\subsection{结果示例}
\begin{itemize}
    \item \textbf{拟合参数 (Cauchy 模型)}:  
    $A = 3.3278,\; B = -24.6992 \,\si{\micro m^2},\; C = 139.1328 \,\si{\micro m^4}$。
    \item \textbf{拟合质量}:  
    $R^2 = -0.3135,\; RMSE = 0.8813$。
    \item \textbf{厚度计算}:  
    \[
    d \approx 14.7 \,\si{\micro m}, \quad
    \Delta \lambda \approx 0.19 \,\si{\micro m}, \quad
    \lambda \approx 3.48 \,\si{\micro m}, \quad
    n \approx 2.18
    \]
\end{itemize}

\subsection{结论}
相较于单一的 Cauchy 模型,Cauchy+Drude 模型能更好地描述掺杂 SiC 外延层在红外波段的折射率变化。实验结果验证了该方法在 \SI{15}{\degree} 入射角条件下的适用性,厚度计算结果具有较高可靠性。
