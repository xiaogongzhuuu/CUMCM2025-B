\section{SiC 外延层厚度计算程序说明}

\subsection{研究背景}
碳化硅 (SiC) 外延层的厚度是决定器件性能的重要参数之一。常用的红外干涉法利用外延层与衬底界面反射形成的干涉条纹,通过分析干涉条纹间隔与折射率的关系,建立厚度数学模型并计算。

然而,SiC 外延层的折射率不是常数,它与波长以及掺杂载流子浓度相关。因此,在传统 Cauchy 色散模型基础上,需要引入 Drude 模型修正,以反映自由载流子对折射率的影响。本文采用 \textbf{Cauchy+Drude 联合模型} 来拟合折射率并计算厚度。

\subsection{数学模型}

\subsubsection{菲涅尔反射率}
在入射角 $\theta_i$ 下,反射率由菲涅尔公式给出:
\begin{equation}
R = \frac{1}{2}\left( 
\left|\frac{\cos\theta_i - n \cos\theta_t}{\cos\theta_i + n \cos\theta_t}\right|^2 
+ 
\left|\frac{n\cos\theta_i - \cos\theta_t}{n\cos\theta_i + \cos\theta_t}\right|^2 
\right),
\end{equation}
其中 $\theta_t$ 由 Snell 定律计算:
\begin{equation}
\sin\theta_t = \frac{\sin\theta_i}{n}.
\end{equation}

\subsubsection{折射率模型 (Cauchy+Drude)}
折射率的平方写作:
\begin{equation}
n^2(\lambda) = \left(A + \frac{B}{\lambda^2} + \frac{C}{\lambda^4}\right)^2 - \frac{Ne^2}{\varepsilon_0 m^\ast \omega^2},
\end{equation}
其中:
\begin{itemize}
    \item 第一项为 \textbf{Cauchy 模型},描述无掺杂情况下的色散;
    \item 第二项为 \textbf{Drude 修正项},考虑自由载流子浓度 $N$ 对红外折射率的影响。
\end{itemize}

\subsubsection{厚度计算公式 (近似)}
根据干涉条件,相邻条纹间隔 $\Delta \lambda$ 与厚度 $d$ 的关系为:
\begin{equation}
d \approx \frac{\lambda^2}{2 n(\lambda) \cos\theta_t \, \Delta \lambda},
\end{equation}
其中 $\lambda$ 取条纹的中值波长。

\subsection{程序实现步骤}

\begin{enumerate}
    \item \textbf{数据读取与预处理}  
    使用 \verb|pandas.read_excel()| 读取实验反射率光谱,并利用 \verb|savgol_filter| 进行噪声滤波;波数转换为波长 (μm)。

    \item \textbf{折射率反演与拟合}  
    利用菲涅尔公式与数值优化 (\verb|minimize_scalar|) 从反射率反演折射率;  
    在有效范围内拟合 Cauchy 参数 $(A, B, C)$;  
    引入 Drude 模型,反演出最佳自由载流子浓度 $N$。

    \item \textbf{条纹提取与厚度计算}  
    使用 \verb|find_peaks| 提取光谱干涉峰位,计算相邻条纹波长间隔 $\Delta \lambda$;  
    取中值波长 $\lambda_{\text{center}}$,代入折射率模型与干涉公式计算外延层厚度。

    \item \textbf{结果输出}  
    输出 Excel 表格与 txt 文件,包括:
    \begin{itemize}
        \item 折射率与波长表格:\verb|波长_折射率_10度.xlsx|
        \item 条纹位置与条纹间隔表格:\verb|条纹间隔_10度.xlsx|
        \item 厚度结果:\verb|厚度计算结果_10度.txt|
    \end{itemize}
    并绘制拟合结果图像。
\end{enumerate}

\subsection{输出结果示例}

程序输出的典型结果如下:
\begin{itemize}
    \item 拟合参数 (Cauchy 模型):  
    $A = 2.874231,\; B = -16.384751 \,\mu\text{m}^2,\; C = 120.983442 \,\mu\text{m}^4$。
    \item 拟合质量:  
    $R^2 = 0.4285,\; RMSE = 0.0921$。
    \item 厚度与条纹信息:  
    \begin{equation*}
    d \approx 14.82 \,\mu\text{m}, \quad
    \Delta \lambda \approx 0.1923 \,\mu\text{m}, \quad
    \lambda_{\text{center}} \approx 3.476 \,\mu\text{m}, \quad
    n \approx 2.187
    \end{equation*}
\end{itemize}

\subsection{结论与说明}
本文提出的 Cauchy+Drude 模型相比单纯的 Cauchy 模型更符合物理实际,能够反映 SiC 外延层掺杂浓度对折射率的影响。结果表明,载流子效应在红外波段对折射率影响有限,厚度计算主要由色散项决定。本方法在厚度提取上可靠,且通过实验光谱与模型的对比验证了合理性。
