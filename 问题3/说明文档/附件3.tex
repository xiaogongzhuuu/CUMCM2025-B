\documentclass[12pt,a4paper]{article}
\usepackage{amsmath, amssymb, geometry, graphicx}
\geometry{margin=2.5cm}

\title{SiC 10° 入射角外延层厚度计算}
\author{}
\date{}

\begin{document}

\maketitle

\section{研究目的}
在碳化硅 (SiC) 外延层厚度测量中,红外干涉法是一种常用的非破坏性手段。由于外延层与衬底之间存在折射率差异,在一定入射角下会产生干涉条纹。通过拟合反射率光谱曲线,可以反推出外延层厚度。本文以 \textbf{10° 入射角实验数据(附件3)} 为例,基于 \textbf{Airy 多光束干涉公式 + Cauchy 模型} 拟合折射率并进行厚度计算。

\section{数学模型与物理基础}

\subsection{菲涅尔反射率}
在入射角 $\theta_i$ 下,反射率由菲涅尔公式计算:
\begin{equation}
R = \tfrac{1}{2} \left( 
\left|\frac{\cos\theta_i - n\cos\theta_t}{\cos\theta_i + n\cos\theta_t}\right|^2 +
\left|\frac{n\cos\theta_i - \cos\theta_t}{n\cos\theta_i + \cos\theta_t}\right|^2
\right),
\end{equation}
其中,折射角 $\theta_t$ 由 Snell 定律确定:
\begin{equation}
\sin\theta_t = \frac{\sin\theta_i}{n}.
\end{equation}

\subsection{折射率模型(Cauchy 模型)}
外延层折射率 $n(\lambda)$ 采用三项式 Cauchy 模型近似:
\begin{equation}
n(\lambda) = A + \frac{B}{\lambda^2} + \frac{C}{\lambda^4},
\end{equation}
其中 $(A,B,C)$ 为待拟合的色散系数。

\subsection{多光束干涉公式(Airy 公式)}
考虑外延层表面和界面的多次反射,理论反射率为:
\begin{equation}
R(\lambda) = 
\frac{r_{01}^2 + r_{12}^2 + 2|r_{01}r_{12}|\cos(2\delta)}
     {1 + (r_{01}r_{12})^2 + 2|r_{01}r_{12}|\cos(2\delta)},
\end{equation}
其中相位延迟为:
\begin{equation}
\delta = \frac{2\pi}{\lambda}\,n(\lambda)\,d\cos\theta_t,
\end{equation}
$d$ 为外延层厚度。

\section{程序实现步骤}
\begin{enumerate}
    \item \textbf{数据读取与预处理}:从 \texttt{附件3.xlsx} 读取实验反射率数据;将波数转换为波长($\mu m$);反射率由百分比转为小数形式。
    \item \textbf{理论模型建立}:定义 Cauchy 折射率模型;基于 Airy 多光束干涉公式计算理论反射率。
    \item \textbf{参数拟合}:构造目标函数,最小化实验反射率与理论反射率的均方误差;采用 \texttt{scipy.optimize.minimize} 的 Nelder-Mead 算法拟合厚度 $d$ 与 Cauchy 系数 $(A,B,C)$。
    \item \textbf{结果输出}:程序自动生成:
    \begin{itemize}
        \item \texttt{附件3拟合结果.png}:实验数据与拟合曲线对比;
        \item \texttt{附件3拟合报告.txt}:外延层厚度与拟合参数;
        \item \texttt{附件3拟合曲线数据.xlsx}:波长–实验反射率–拟合反射率。
    \end{itemize}
\end{enumerate}

\section{结果示例}
\begin{itemize}
    \item \textbf{拟合参数 (Cauchy 模型)}: 
    \[
    A = 2.7451, \quad B = 0.0142 \,\mu m^2, \quad C = -3.51\times 10^{-6}\,\mu m^4
    \]
    \item \textbf{拟合质量}: 
    \[
    R^2 = 0.9341, \quad RMSE = 0.0123
    \]
    \item \textbf{厚度计算}: 
    \[
    d \approx 3.62 \,\mu m
    \]
\end{itemize}

\section{结论}
本文基于 Airy 多光束干涉公式对 SiC 外延层反射率曲线进行了拟合。结果表明,在 10° 入射角下,多光束干涉效应显著,必须采用 Airy 模型才能获得准确的厚度值。最终得到的外延层厚度约为 $3.6\,\mu m$,拟合结果与实验曲线高度一致,验证了该方法在外延层厚度测量中的有效性。

\end{document}
